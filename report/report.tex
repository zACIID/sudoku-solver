\chapter{Notes}

\section{Constraint Propagation}

\begin{itemize}
    \item Per prima cosa parlare della teoria del Constraint Propagation, in particolare:
    \begin{itemize}
        \item (intro1) in cosa consiste (utilizzare dei vincoli definiti per ridurre il set di mosse possibili, e dunque lo spazio di ricerca da esplorare)
        \item (intro2) il fatto che sia una buona tecnica da applicare al sudoku, in quanto...
        \item definizioni formali (dominio, vincolo)
        \item come entra in gioco il backtracking? devo spiegare anche quello ma non so a che punto inserirlo
    \end{itemize}
    \item parlare della mia implementazione
    \begin{itemize}
        \item come calcolo il dominio
        \item il fatto che ordino le celle partendo da quelle col dominio minimo
    \end{itemize}
\end{itemize}

\section{Simulated Annealing}

\begin{itemize}
    \item Per prima cosa parlare della teoria del simulated annealing, in particolare:
    \begin{itemize}
        \item perché si chiama così (da dove deriva il nome)
        \item in cosa consiste (intuizione: si va a tentativi finché si trova la strada giusta, se si rimane incastrati si "sale")
        \item il fatto che non sia una procedura indicata per il sudoku, o che quantomeno ne esistono di migliori, tipo CP
    \end{itemize}
    \item parlare della mia implementazione
    \begin{itemize}
        \item il fatto che prima ho provato a mettere numeri a caso e basta
        \item il fatto che poi ho migliorato settando già le righe con i numeri giusti, pur ottenendo risultati scarsi
        \item come definisco la funzione punteggio in entrambi i casi
        \item come setto la temperatura (discorso del link che partenza deve permettere p >= 80\% e che poi scende)
        \item come resetto la temeperatura in caso mi incastro (coeff1 scende in base al tempo, coeff2 sale in base alla stuckness, temperatura no reset più alto di valore iniziale)
    \end{itemize}
\end{itemize}

\section{Conclusions}

Dire che CP è meglio di SimAnn perché sudoku è un problema dai constraint definiti e facilmente implementabili.
SimAnn, invece, per definizione è un metodo che va a tentativi, e in un problema come il sudoku in cui le combinazioni totali
sono molte, e quelle valide ben poche, non è efficace.