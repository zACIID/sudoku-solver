% Packages import
\usepackage[utf8]{inputenc}
\usepackage[english]{babel}  % set english language
\usepackage{graphicx}
\usepackage{geometry}
\usepackage{lipsum}
\usepackage{afterpage}
\usepackage[pagestyles]{titlesec}
\usepackage{titlecaps}
\usepackage{indentfirst}
\usepackage{xurl}
\usepackage{blindtext}
\usepackage{hyperref}
\usepackage{listings}
\usepackage{xcolor}
\usepackage{caption}
\usepackage{placeins}
\usepackage{subcaption}
\usepackage{amsfonts}
\usepackage[fleqn]{amsmath}
\usepackage{pgfplots}
\usepackage{multirow}

% -------------------- FORMATTING --------------------

% Page formatting
\newgeometry{
    top=0.75in,
    bottom=0.75in,
    outer=0.85in,
    inner=0.85in
}

% Update font-size command
\providecommand{\updatefont}
{
  \fontsize{6mm}{6mm}\selectfont
  \titleformat*{\section}{\LARGE\bfseries}
    \titleformat*{\subsection}{\Large\bfseries}
    \titleformat*{\subsubsection}{\large\bfseries}
    \titleformat*{\paragraph}{\large\bfseries}
}

% Force images to not overlap with following section and subsections
\let\Oldsection\section
\renewcommand{\section}{\FloatBarrier\Oldsection}

\let\Oldsubsection\subsection
\renewcommand{\subsection}{\FloatBarrier\Oldsubsection}

\let\Oldsubsubsection\subsubsection
\renewcommand{\subsubsection}{\FloatBarrier\Oldsubsubsection}

% Graphs backward compatibility
\pgfplotsset{compat=1.18}

\renewcommand\labelitemi{$\vcenter{\hbox{\tiny$\bullet$}}$}

% -------------------- STYLE --------------------

% Title format: Chapter number + chapter name
\titleformat{\chapter}{\normalfont\huge \bfseries}{\Huge \thechapter}{20pt}{\Huge} 

% Chapter number starts from 0 (Introduction to the assignment)
\setcounter{chapter}{-1}

% Numbering subsubsection
\setcounter{tocdepth}{3}
\setcounter{secnumdepth}{3}

% Hyperlinks color
\definecolor{urlcolor}{HTML}{010a47}
\hypersetup{
    colorlinks=true,    
    linkcolor=black,
    citecolor=urlcolor,
    urlcolor=urlcolor
}

% -------------------- UTILS --------------------

% Empty page
\newcommand\blankpage{%
    \null
    \thispagestyle{empty}%
    \addtocounter{page}{-1}%
    \newpage}
    
% Title page infos
\providecommand{\info}[2]
{
  \noindent
    {\bf{#1}\\}
    {#2\\}
    \vspace{0.22mm}
}

% Keywords command
\providecommand{\keywords}[1]
{
  \small	
  \textbf{\textit{Keywords---}} #1
}

% JSON

\definecolor{jsonkey}{HTML}{B03A2E}
\definecolor{jsonvalue}{HTML}{1F618D}
\definecolor{jsonnumber}{HTML}{229954}

\newcommand\jsonkey{\color{jsonkey}}
\newcommand\jsonvalue{\color{jsonvalue}}
\newcommand\jsonnumber{\color{jsonnumber}}

% switch used as state variable
\makeatletter
\newif\ifisvalue@json

\lstdefinelanguage{json}{
    tabsize             = 4,
    showstringspaces    = false,
    keywords            = {false,true},
    alsoletter          = 0123456789.,
    morestring          = [s]{"}{"},
    stringstyle         = \jsonkey\ifisvalue@json\jsonvalue\fi,
    MoreSelectCharTable = \lst@DefSaveDef{`:}\colon@json{\enterMode@json},
    MoreSelectCharTable = \lst@DefSaveDef{`,}\comma@json{\exitMode@json{\comma@json}},
    MoreSelectCharTable = \lst@DefSaveDef{`\{}\bracket@json{\exitMode@json{\bracket@json}},
    basicstyle          = \ttfamily\small,
    breaklines          = true
}

% enter "value" mode after encountering a colon
\newcommand\enterMode@json{%
    \colon@json%
    \ifnum\lst@mode=\lst@Pmode%
        \global\isvalue@jsontrue%
    \fi
}

% leave "value" mode: either we hit a comma, or the value is a nested object
\newcommand\exitMode@json[1]{#1\global\isvalue@jsonfalse}

\lst@AddToHook{Output}{%
    \ifisvalue@json%
        \ifnum\lst@mode=\lst@Pmode%
            \def\lst@thestyle{\jsonnumber}%
        \fi
    \fi
    %override by keyword style if a keyword is detected!
    \lsthk@DetectKeywords% 
}