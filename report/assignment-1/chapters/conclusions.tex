\chapter*{Final Considerations}

The two approaches implemented for this assignment are very different in nature: \textit{Constraint Propagation} is based on the exploitation of a set of well defined rules to reduce the search space and greedily come to the final solution, while \textit{Simulated Annealing} is a stochastic process that tries to optimize some type of score, the energy, by performing some type hill-climbing, with the added characteristic of being able to overcome local minima worse than the global one. As such, the former approach is the best suited to solve the Sudoku problem, since the set of constraints is simple and well defined. The latter, instead, while being almost always able to find a solution given a reasonable computational budget, tends to suffer, at least when compared to the Constraint Propagation approach, because of the combinatorial aspect of the problem, which makes the search space big and hence relatively difficult to explore in a stochastic manner.
\par
Empirical testing has shown that the \textit{Backtracking, Constraint Propagation} solver always outperforms the \textit{Simulated Annealing} one, thereby proving that the latter is ill-suited to solve the Sudoku problem.